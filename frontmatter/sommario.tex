\begin{abstract}[it]
    La gestione dei dati biomedici è sempre più complessa a causa della varietà dei sistemi di archiviazione e dei modelli di dati in evoluzione. Questa eterogeneità presenta ostacoli all'integrazione e alla consultazione dei dati, cruciali per il progresso della ricerca biomedica e dell'assistenza sanitaria. Il progetto comporterà la progettazione di un'architettura di sistema che supporti l'integrazione di archivi di dati eterogenei sotto un sistema federato unificato. Il sistema utilizzerà anche i principi del paradigma dell'Accesso ai Dati Basato su Ontologie (\ac{OBDA}) per facilitare le capacità di consultazione semantica. Sviluppando un sistema federato di analisi dei dati per i dati genomici, questa tesi contribuirà a ridurre le complessità coinvolte nella gestione dei dati biomedici. Il sistema permetterà processi di integrazione e consultazione dei dati più efficaci, supportando così una ricerca genomica più rapida e accurata. Il completamento di questo progetto porterà alla realizzazione di un prototipo di un sistema federato di analisi dei dati capace di gestire le esigenze specifiche dei dati genomici.
\end{abstract}