%!TEX root = ../main.tex

\chapter{Conclusions and Future Works}
\label{chp:conclusions}

The research and development work undertaken in this thesis has focused on the challenging task of integrating and analyzing heterogeneous biomedical data, specifically clinical and genomics data. The primary aim has been to design a robust and scalable federated data analytics system, capable of handling the complexities associated with diverse data sources. Leveraging the \ac{OBDA} paradigm and advanced data federation techniques, the system provides a unified platform that not only integrates but also semantically enriches the data from multiple, disparate sources.
The system's architecture, built on Dremio as the data federation layer and Ontop for semantic data integration, has demonstrated the feasibility and effectiveness of using these technologies in a federated environment. The choice of Dremio was driven by its open-source nature, robustness, scalability, and ability to manage different types of data sources, including relational databases, polystore systems, and cloud storage solutions. Ontop was selected as well because it is \ac{FOSS}, but also for its compliance with \ac{W3C} standards and its capability to perform high-performance query answering over virtualized \ac{RDF} graphs. Together, these technologies compose the backbone of the proposed system, enabling complex queries over diverse datasets without the need for extensive data preprocessing or manual data integration.
The implementation of this system within the context of the \ac{HEREDITARY} project has further validated its applicability in real-world scenarios. The federated data analytics system developed in this thesis has been designed as an initial prototype that addresses the specific needs of this project, facilitating the integration and analysis of clinical and genomics data from different medical centers across Europe. The system's ability to handle both structured and unstructured data, its support for real-time data retrieval, and its compliance with privacy regulations such as \ac{GDPR}, make it a valuable tool for biomedical research.
However, the work presented in this thesis is not without limitations, and several areas for future research and development have been identified. One of the most promising directions for future work is the packetization or dockerization of the system's software components, so to transform it into a single reusable application. By containerizing the various components of the system, such as the Dremio and Ontop instances, it would be possible to simplify the deployment process and ensure consistent performance across different environments. Dockerization would also facilitate the scaling of the system, allowing it to handle larger datasets and more complex queries by distributing the workload across multiple containers.
Another area for future research is strengthening the system's compliance with \ac{GDPR} and other data protection regulations. While the current system has been designed with privacy and data security in mind, there is always room for improvement. Future work could focus on developing more sophisticated techniques for data anonymization and encryption, as well as implementing stricter access controls and auditing mechanisms to ensure that sensitive data is always protected. Additionally, further research could explore the integration of federated learning techniques, which would allow for the analysis of data across multiple sites without the need to share the raw data itself, thus enhancing privacy and security.
Optimizing the system's performance is another critical area for future work. The benchmarking results presented in this thesis have highlighted the strengths and weaknesses of the current architecture, particularly in terms of query execution time and resource consumption. Future research could focus on refining the system's performance by utilizing the results of the benchmarking as a ground point for optimization. This could involve fine-tuning the system's configuration, improving the efficiency of the query rewriting and unfolding processes, or developing new algorithms for data federation and virtualization. Special attention should be given to optimizing the system's performance during peak loads, as these are often the most critical points in terms of resource usage and response time.
In addition to these specific areas of future research, there are also broader questions that could be explored in relation to the system's overall design and functionality. For example, the system be made more user-friendly, particularly for researchers and clinicians who may not have a background in data science. As an example, it could be possible to develop a custom endpoint, developed from scratch, that interfaces with the architecture from the top, allowing to easily and visually build queries to deliver to the federated architecture, and that effectively allows to perform various analytics based on retrieved data.
In conclusion, the federated data analytics system developed in this thesis represents a significant step forward in the integration and analysis of heterogeneous biomedical data. When combining advanced data federation techniques with the semantic power of \ac{OBDA}, the system offers a scalable, flexible, and privacy-conscious solution for managing the complexities of clinical and genomics data. As said, the work is far from complete, and there are many opportunities for future research and development. Continuing to build on the foundations laid in this thesis will make it possible to create even more powerful and versatile tools for biomedical research, contributing to better healthcare outcomes and a deeper understanding of human health and disease.