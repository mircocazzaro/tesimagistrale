%!TEX root = ../main.tex

\chapter{Introduction}
\label{chp:intro}

In recent years, the increasing complexity of biomedical data has posed significant challenges to researchers and clinicians. The rapid evolution of data storage systems and the diversity of data models have contributed to these challenges, creating a heterogeneous landscape that is difficult to navigate. This complexity is particularly pronounced in the field of genomics, where the integration and analysis of diverse datasets are crucial for advancing our understanding of genetic diseases and developing personalized medical treatments.
The advent of big data analytics has led to new possibilities for handling large volumes of complex biomedical data. However, the vast diversity of data sources (e.g. relational databases, hierarchical storage systems and graph-based models) necessitates the development of sophisticated data integration frameworks. These frameworks must not only accommodate the different data models but also enable seamless querying and analysis across these models. The need for such frameworks is especially crucial in genomics, where the ability to integrate and analyze data from multiple sources can significantly accelerate research and improve clinical outcomes.
The focus of this thesis is on the design and implementation of a federated data analytics system specifically tailored for the integration of clinical and genomics data. This system is designed to address the challenges associated with integrating heterogeneous data sources, enabling researchers to perform complex queries across multiple datasets without the need for extensive data preprocessing or manual data integration. By leveraging the principles of \ac{OBDA}, the proposed system eases semantic querying capabilities, allowing users to extract meaningful insights from the data more efficiently.
\ac{OBDA} is a powerful paradigm for data integration, particularly in environments characterized by data heterogeneity. Although research activities on OBDA started more than 20 years ago, it is still nowadays a subject of discussion in the academic field as well as in the industry, having periodically novel papers discussing both new frameworks or their improvements and their employment in industrial scenarios. OBDA allows for the seamless integration of relational databases into an ontology framework, enabling users to perform semantic queries that transcend the limitations of traditional data retrieval methods. The use of ontologies in OBDA provides a shared vocabulary and a formalized structure for representing knowledge within a specific domain, which is particularly beneficial in the field of genomics where data is often complex and highly interconnected.
The system developed in this thesis builds upon the OBDA paradigm by integrating it into a federated data architecture. This architecture is designed to support the integration of multiple heterogeneous data sources, including relational databases, NoSQL systems, and cloud storage solutions. By creating a unified federated system, the architecture allows for real-time data retrieval and integration, eliminating the need for data deduplication and ensuring that researchers have access to the most current data available.
A key component of this system is the use of a specialized ontology that models the intricate relationships between genomic data and clinical information. The ontology serves as the backbone of the system, enabling the semantic integration of data from diverse sources and facilitating complex queries that would be difficult or impossible to perform using traditional data retrieval methods. The ontology's design is informed by the specific requirements of clinical and genomics research, with a focus on ensuring interoperability and scalability as new data sources and data types are introduced.
The federated architecture proposed in this thesis also incorporates advanced data federation techniques, which are essential for managing the diversity of data sources in genomics. Data federation allows for the creation of a virtual data access layer that abstracts the underlying technical details of each data source, enabling researchers to query data using standard languages like SQL without needing to know where the data is physically stored or in what format. This approach not only simplifies the data retrieval process but also minimizes the risks associated with data movement and duplication.
Furthermore, the system is designed with scalability in mind, allowing it to accommodate new data sources as they become available. The system's architecture is flexible enough to integrate these new data sources seamlessly, ensuring that researchers can always work with the most comprehensive dataset possible.
Another critical aspect of the system is its focus on privacy and data security. Given the sensitive nature of clinical and genomic data, especially when data is strictly related to patients’ personal information, the system aims to adhere to modern data protection regulations, such as the \ac{GDPR}. This is to ensure that while data is integrated and analyzed, it remains protected and secure, ensuring patient privacy and maintaining the trust of the institutions and individuals who provide the data.
The thesis also explores the practical application of the proposed system within the context of the HEREDITARY project, a European Union-funded initiative focused on integrating multimodal data to advance the understanding of brain diseases. The HEREDITARY project represents a real-world application of the federated data analytics system, demonstrating its potential to facilitate complex analyses and drive new insights in biomedical research.
Finally, an extensive benchmarking phase will evaluate the performances of the proposed architecture considering both its strengths as a \ac{DBMS}, analyzing aspects as average execution time and throughput, and the resource consumption of the system, in order to figure out whether such an application adoption may be feasible across diverse hardware environments.
In summary, this thesis presents a comprehensive approach to addressing the challenges of integrating and analyzing heterogeneous clinical and genomics data. Posing its foundations on the OBDA paradigm and advanced data federation techniques, the proposed system offers a robust and scalable solution for managing the complexities of biomedical data.
