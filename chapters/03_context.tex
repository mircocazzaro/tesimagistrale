%!TEX root = ../main.tex

\chapter{Context of the Work}
\label{chp:context}
Handling biomedical data entails considerable challenges. Practitioners have to deal with multiple distinct storage systems, each using different data models, that may be subject to evolution over time. This introduces substantial heterogeneity, with the same data domain represented possibly through various models, such as relational \cite{DBLP:journals/nar/HardingAFSADPSD22}, hierarchical \cite{DBLP:journals/nar/WishartKGSHSCW06}, or graph based. Graph models are particularly prevalent in biomedical data \cite{DBLP:journals/nar/PineroBQGDCGSF17} \cite{DBLP:journals/nar/GillespieJSMRSG22}, as their structure effectively represents the intricate and interconnected relationships characteristic of biological systems.
These systems typically operate under various database management systems (DBMSs). Further, the data is described using diverse metadata schemas, increasing the heterogeneity. This fragmentation complicates the ability to query these systems together and infer new knowledge, unless experts manually integrate them, by defining a unified data model, achieving consensus among data stakeholders, manually matching existing data to this new model, migrating data accordingly, and then modifying applications to adapt to changes in the used query language. These steps are both time-consuming and expensive.
At the European level, there are many contexts where these challenges are relevant. Therefore, in this context, we can actively pursue our objectives, grappling with collections of unstructured and heterogeneous biomedical data. In particular, the Department of Information Engineering at the University of Padova leads the EU project \ac{HEREDITARY}\footnote{https://\ac{HEREDITARY}-project.eu/}, collaborating with three medical centers that manage heterogeneous multimodal clinical and genomic data requiring integration. \ac{HEREDITARY} emphasizes the critical need for a capable and efficient system that can seamlessly integrate diverse biomedical datasets. By setting up an effective federated architecture, we aim to automate the querying process across various data models, cutting the cost of performing analytics. This approach will facilitate a more detailed analysis of the clinical and genomic data from the three medical centers participating in \ac{HEREDITARY}.
In this chapter, we will discuss the overall structure of the \ac{HEREDITARY} project, and we will discuss the structure of available clinical and genomics data. By this discussion we will highlight which features are of interest for a federated data analytics platform that has to manage efficiently these sources of data.

\section{The HEREDITARY Project}

The \ac{HEREDITARY} project represents a groundbreaking initiative funded by the European Union, aimed at transforming our understanding of brain diseases through an innovative convergence of multimodal heterogeneous data, such as genomic data, bioimages, clinical records and environmental data. This ambitious project is structured to harness the power of big data and advanced analytics to tackle some of the most pressing challenges in healthcare today.
At the heart of the \ac{HEREDITARY} project is its commitment to change the paradigm of healthcare by integrating diverse data streams to unlock previously inaccessible insights. This integration involves sophisticated data linkage across various modalities. By leveraging this integrated data framework, \ac{HEREDITARY} seeks to revolutionize the fields of disease detection, treatment response, and preventive healthcare.
\subsubsection{Privacy Compliance}
Security and privacy compliance stands at the fundamentals of the \ac{HEREDITARY} project, particularly in handling sensitive health data. The project employs state-of-the-art secure supercomputing facilities and federated learning techniques. These methodologies ensure that while the data is extensively analyzed to yield critical health insights, it remains within the confines of local data governance laws, such as the \ac{GDPR}. This system not only protects individual privacy but also facilitates a collaborative environment where data does not cross organizational boundaries unnecessarily.
\subsubsection{Technological Components}
\ac{HEREDITARY} is pioneering the use of advanced learning models, including deep neural networks and self-supervised learning algorithms, to analyze large sets of heterogeneous data. The project's focus on semantic-aware learning methodologies, facilitated by the \ac{OBDA} paradigm, allows for the seamless integration of disparate data types. These integrated datasets are utilized to perform complex queries and enhance predictive analytics capabilities, which are crucial for identifying new disease patterns and treatment possibilities.
\subsubsection{User-Friendly Analytical Platforms}
To maximize the impact of its research findings, \ac{HEREDITARY} is developing interactive data-driven solutions that simplify the exploration and analysis of complex health data. The project includes the creation of a visual analytics platform that combines advanced data visualization tools with user-friendly interfaces. This initiative not only aids researchers and clinicians in hypothesis testing and decision-making but also enhances public understanding and trust in health data use.
\subsubsection{Collaborativity and Multidisciplinarity}
\ac{HEREDITARY}'s structure is inherently collaborative, involving multiple leading European universities and research institutions. Each participant brings unique expertise and resources, facilitating a comprehensive approach to tackling healthcare challenges. The project encourages continuous interaction among all partners, ensuring that insights and methodologies are shared and refined across different disciplines and sectors.
\subsubsection{Focus on Neurodegenerative and Gut-Related Disorders Interplay}
One of the primary research focuses of the \ac{HEREDITARY} project is the investigation of neurodegenerative and gut microbiome-related disorders. By examining the gut-brain axis and its impact on diseases such as Parkinson’s and Alzheimer's, the project aims to uncover novel therapeutic and diagnostic approaches that could lead to more personalized medicine practices. These efforts are supported by sophisticated data models that predict disease progression and treatment responses, tailoring healthcare strategies to individual patient needs.
\subsubsection{Future Perspectives}
\ac{HEREDITARY} is set to continue its influence beyond the project's timeline by developing sustainable strategies for health data utilization. The project's outputs are expected to inform future policy, enhance clinical practices, and continue to provide valuable insights into complex health conditions. By establishing a robust framework for data integration and analysis, \ac{HEREDITARY} positions itself as a beacon for future initiatives in the realm of data-driven healthcare innovation.
\subsubsection{\ac{HEREDITARY} Project Structure}
The duration of the \ac{HEREDITARY} project is four years. The structural organization of the \ac{HEREDITARY} project is meticulously designed to ensure efficient project management, seamless collaboration across disciplines, and rigorous pursuit of its scientific objectives. The project is divided into nine distinct \ac{WP}, each tasked with specific aspects of the project's implementation and goals.