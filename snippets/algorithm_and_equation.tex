\begin{algorithm}[ht]
    \caption{An algorithm with caption}\label{alg:two}
    \begin{algorithmic}
        \REQUIRE $n \geq 0$
        \ENSURE $y = x^n$
        
        \STATE $y \gets 1$
        \STATE $X \gets x$
        \STATE $N \gets n$
        
        \WHILE{$N \neq 0$}
            \IF{$N$ is even}
              \STATE $X \gets X \times X$
              \STATE $N \gets \frac{N}{2} $  \COMMENT{This is a comment}
            \ELSIF{$N$ is odd}
              \STATE $y \gets y \times X$
              \STATE $N \gets N - 1$
            \ENDIF
        \ENDWHILE
    \end{algorithmic}
\end{algorithm}

\begin{equation}
e^{j\pi} + 1 = 0
\end{equation}

\begin{lstlisting}[language=Python, caption=Code snippet example]
  import numpy as np
      
  def incmatrix(genl1,genl2):
      m = len(genl1)
      n = len(genl2)
      M = None #to become the incidence matrix
      VT = np.zeros((n*m,1), int)  #dummy variable
  
      test = "String"
      
      #compute the bitwise xor matrix
      M1 = bitxormatrix(genl1)
      M2 = np.triu(bitxormatrix(genl2),1) 
  
      for i in range(m-1):
          for j in range(i+1, m):
              [r,c] = np.where(M2 == M1[i,j])
              for k in range(len(r)):
                  VT[(i)*n + r[k]] = 1;
                  VT[(i)*n + c[k]] = 1;
                  VT[(j)*n + r[k]] = 1;
                  VT[(j)*n + c[k]] = 1;
                  
                  if M is None:
                      M = np.copy(VT)
                  else:
                      M = np.concatenate((M, VT), 1)
                  
                  VT = np.zeros((n*m,1), int)
      
      return M
  \end{lstlisting}

  
\begin{figure}[ht]
    \centering
    \includegraphics[width=0.5\textwidth]{res/ltunipd}
    \caption{Example of image}
\end{figure}